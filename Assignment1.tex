\documentclass[a4paper]{llncs}

\usepackage{graphicx, verbatim, tabularx, cite, url, textcomp}
\usepackage{array}
\usepackage[margin=2.5cm]{geometry}

\title{Security in Organizations - Assignment 1}
\author{Wouter de Groot [s4336070] and Erik Schneider [s4404149]}
\institute{Technische Universiteit Eindhoven\\
Eindhoven, The Netherlands\\
October 6, 2014}

\begin{document}	
\maketitle

\section{Part 1 - Malware}
\subsection{A. Type Classification}
\subsubsection{Kinds of malware}
Researchers disagree on the terms and classification of malware but Stallings  \cite{stallings2007network} has created a “Terminology of Malicious Programs” that we have included as fig. \ref{fig:malwaretable}. Broadly, a program can be entirely malicious, be made partially malicious either at the design stage or by a third party, or be a platform to attack legitimate programs.\\

The first category includes downloaders, keyloggers, rootkits and similar. These are pieces of software specifically crafted to perform an (automated) attack, and nothing else. Examples include Slammer which automatically propagated through a buffer overflow in Microsoft’s SQL Server and the ILOVEYOU virus which used social engineering by masquerading as a love letter.\\

In the partially malicious group are programs to which either an external party has attached malicious payload, such as a trojan horse being integrated in a free game available for download, or a software vendor has modified. Perhaps they included some method of gaining access or functionality outside of the normal operating parameters--dubbed a backdoor--or they hid a payload behind a set of conditions to be met, perhaps by remote activation. The latter is called a logic bomb.\\

Finally, similar to the inherently malicious programs are attackers. These are programs or snippets of code which an attacker wields against legitimate software. These can be DoS tools, exploits like SQL injection or XSS strings (or automated programs) or similar. Depending on their application, one might consider Metasploit and their ilk malicious, too. This last point highlights an issue also seen in discussions of physical weaponry. In the right hands attack tools are used to analyze an organisation and to strengthen its defense, but when attackers use them instead the results can be particularly damaging.\\

\begin{figure}
    \centering
    \includegraphics[scale=.35]{malwaretable}
    \caption{Figure of Stallings’ “Terminology of Malicious Programs”, p. 342.}
    \label{fig:malwaretable}
\end{figure}

\subsubsection{Methods of infection}
Delivery mechanisms are also mentioned in the descriptions of fig. \ref{fig:malwaretable} and can also be used to classify malware. One can distinguish between automatically propagating code (essentially just worms as classified by the table, even though the actual attack vector might be an exploit or a backdoor) and code requiring user intervention. Usually, this intervention is deployed as some kind of social engineering, such as attaching a legitimate looking but malicious PDF to an email, offering a free game or any other scheme where users might want to open or execute the content. Email, USB drives, web pages, vulnerable services being offered and exchanging floppy disks are additional examples of delivery mechanisms.\\

\subsubsection{Types of misuse}
Once an attacker has infected a host, the options for misuse are nearly endless (assuming they managed to gain system level access). She could use the host as a proxy to attack other hosts, use the bandwidth available to it for DDoS attacks, store illicit material on the machine or even make it mine BitCoins. If the host is used by people, then the attacker could steal their files, record their behavior and post videos online, log their keystrokes, manipulate their online accounts and profiles, encrypt their hard drives for ransom, or delete everything.\\

In short, anything a legitimate owner or operator of the machine could do, the attacker can now do, too. Even when an attacker only obtains the privileges of a limited account she still likely has access to some interesting data or resources. As an especially damaging case, if the privileges belong to one of the principal users of the host, then all of this user’s private information will still be vulnerable to extraction.

\subsection{B. personalRiskAssessmentAndMitigation}
In this section we first discuss individuals in general and the various measures they can take, and then explain what we, the authors, actually do. First, however, we note that the effects of malware to individuals overlap strongly with the possible effects organisations face: loss of control over hardware resources, data leakage, compromise of (financial) account information, loss of privacy. Individuals do likely have fewer interesting assets, and for that reason the threat model is simplified with less emphasis on tailor-made attacks and more on phishing campaigns, fake websites with exploits or malicious downloads and similar threats which cast a wide but inspecific net.

\subsubsection{Preventive Measures}
Malware infection can be considered a failure in access control. Passwords are one form of access control, and their proper management improves data security. User and root passwords should be sufficiently strong, a measure that increases through time in proportion to processing power. They should not be reused elsewhere to prevent one password leak from affecting multiple assets. Although password managers can help users achieve both objectives, they also represent a single point of failure and researchers have identified weaknesses in the security of these services \cite{zhao2013vulnerability}.\\

Software, both operating system and applications, should be kept up-to-date with automatic patching services enabled where available. To aid in this, users may wish to subscribe to security-announce mailing lists. Web browsers provide extensions and plug-ins that reduce the risk of phishing websites by disabling Javascript by default. Users may also use alternate DNS nameservers, such as Google Public DNS, that employ DNSSEC and protect against phishing attacks. In addition, they can run anti-virus and anti-spyware applications on all devices that are network enabled. These applications provide basic protection but researchers have questioned the overall efficacy of these products \cite{goodin2014avdead}.\\

Sandboxing is another malware mitigation technique. Modern browsers contain logic that prevents malware from accessing the rest of the system and stops browser tabs from affecting one another. Firewalls regulate packet flows into and out of a network or device. A properly configured firewall can provide strong security but the required expertise to configure one correctly is not trivial.\\

Individuals can mitigate against the potential effects of malware by ensuring redundancy in devices by having, for example, a tablet as well as a laptop and in data by making regular backups. Preventing access to sensitive information can be performed on multiple levels. One can lock work stations, lock doors, encrypt data, and require multiple authentication factors.\\

Procedural measures also exist to prevent or mitigate damage by malware. In the US, for example, one can ‘freeze’ credit reports with the major credit bureaus. This prevents credit bureaus from releasing credit data to inquiring parties until permission is explicitly granted. Thus an attacker who has stolen someone’s identity information cannot automatically open new lines of credit in that person’s name.

\subsubsection{Detective Measures}
Several detective measures exist to alert individuals that their data has been compromised. Credit services, such as banks, offer fraud notifications based on unusual account activity. Customers are also advised to regularly audit their account records. Anonymous behavior of an application, a network or a website can also alert individuals that malicious processes are running. The level of sophistication in detecting such behavior can range from intuition (all your money is gone) to free software (e.g. Snort) to commercial intrusion detection systems. On a more technical level, the aforementioned antivirus and firewall systems generally include logging and alerting facilities as well. Some operating systems give a warning whenever unsigned binaries are about to be executed on the system. GateKeeper is such a function on Mac OS X.

\subsubsection{Corrective measures}
Once malware is detected, or even suspected, a user can take several corrective measures. Among the first should be to isolate the infected machine and attempt to remove the malware. If the malware cannot be verifiably removed or has corrupted the data and services sufficiently, then one should restore the system to a prior state using backups. An analysis of how the malware entered the system can inform the adoption of future countermeasures. When sensitive credit, financial, or identifying information may have been exfiltrated, one should change possibly affected passwords and notify pertinent financial, credit, and government services that a breach has occurred.

\subsubsection{Residual Risk}
Several factors determine one’s residual risk to malware but it’s directly proportional to one’s overall investment in security measures and procedures. Trusted relations, such as family members, that have access to network enabled devices and resources may not follow best practices and/or downgrade security measures for convenience. For example, a child may download games from a malicious site despite system warnings.\\

The networked individual can never eliminate the threat of malware, however. Modern security tools cannot safeguard against all zero-day attacks and advanced persistent threats. Furthermore, people can be indirectly exposed to malware because their data is widely distributed across many entities and vendors. Malware that extracts sensitive data from a commercial billing server has the same effect as if it was exfiltrated from a user’s personal computer.\\

\subsubsection{Personal situation}
Although security through obscurity is bad design, providing information for free to potential attackers is not a good idea, either. A somewhat snarky response therefore is to note that one part of our personal security stance is not to reveal details about our personal security. In the interest of this assignment we will make exceptions.\\

We keep up with industry developments in general and software updates specifically and deploy security updates when available. Our internal networks are regulated by firewalls. We use systems of redundant backups both on and off-site to guard against hardware failure, physical calamity, theft and malware. Where possible we restrict code execution to signed binaries: repositories in Linux, GateKeeper in Mac OS X, and respective app stores on mobile systems. We lock our computers whenever they are unattended, and lock the doors. Only people we trust are allowed to be unattended in our places of residence. The aforementioned firewalls and backup systems send us email when irregularities occur.

\subsubsection{Future Trends}
The continued shift towards mobile devices and bring-your-own-device means that the location of data, as well as the logical position of devices in a network is more fluid than ever. At the same time, since the lines between work and private life are blurring, corporate computers might find themselves in employees’ homes. This intermingling poses a unique challenge to security architects. Defense in depth becomes harder to setup when malware can enter an organisation via a compromised corporate laptop that got attacked by the employee’s home router. Advanced persistent threats will increasingly use these novel attack vectors to perform surgical strikes against high-value targets. Stuxnet is an early example of this type of tailor-made attack<cite something on Stuxnet? maybe http://www.langner.com/en/wp-content/uploads/2013/11/To-kill-a-centrifuge.pdf>.\\

Could cite Langner here:
“At the operational level, Stuxnet highlighted the royal road to infiltration of hard targets. Rather than trying to infiltrate directly by crawling through fifteen firewalls, three data diodes, and an intrusion detection system, the attackers played it indirectly by infecting soft targets with legitimate access to Ground Zero: Contractors. Whatever the cyber security posture of contractors may have been, it certainly was not at par with the Natanz Fuel Enrichment facility. Getting the malware on their mobile devices and USB sticks proved good enough as sooner or later they would physically carry those on site and connect them to the FEP’s most critical systems, unchallenged by any guards.“

The arms race between hackers and defenders has no end in sight, and it will affect both organizations and private individuals. The Internet of Things will increase the attack surface for everyone and the constrained nature of embedded devices may limit security protections. There are promising areas in security research, however. Microkernels and powerful correctness provers may one day significant reduce malware damage but their widespread adoption has not occurred, and legacy systems will continue to present problems. We’re also seeing a stronger focus on encryption (Eric Holder is not amused) and chains of trust (cf. WinPhone), which is encouraging. 


\section{DigiNotar}
\subsection{C. DigiNotar Case Analysis}
The Black Tulip report from Fox-IT appears to be rather thorough. While it is unsatisfactory that the research effort eventually veered away from retracing the attacker’s steps, the need at the time to ascertain the compromise of the Certificate Authorities makes it understandable.\\

The takeaway from the report--other than the lessons learned--is that DigiNotar had an almost nonexistent security monitoring and response infrastructure. It is inappropriate that it took more than a week for personnel to notice that certificate generation logs did not match the administrative record. It is likewise difficult to believe that the net-facing components of DigiNotar’s infrastructure used outdated software versions with known vulnerabilities. Taken in its entirety, the Black Tulip report paints a picture of a deeply dysfunctional security service provider. One may go so far as to observe that the prime lesson learned from the case analysis is simply that DigiNotar should have implemented ISO27001. Their failure to do so combined with an advanced persistent threat means it is no surprise DigiNotar did not survive.\\

Their manifold security lapses paint a grim picture for the system of CAs in general. Consider the list of trusted certificates available on iOS \cite{apple2014list}. If any of the CAs in this list suffer a similar compromise to the one at DigiNotar, hundreds of millions of users worldwide would be put at risk. The situation is no different for any other operating system or web browser.\\

\centering{\textbf{Security Management Lessons for All Organizations\footnote{We found the wording of assignment 2C to be ambiguous. First, it was not clear if we were to ignore Fox-IT’s well thought out lessons or create new ones. Second, we were unsure whether to separate the lessons into CA/CA customer groups or technical/security management groups (or both). The following reflects our choices}}}
\begin{description}
  \item[Complement prevention with detection / defense in depth] \hfill \\
Organisations should deploy a layered security model consisting of multiple and varied obstacles between attackers and assets (prevention). It also should consist of measures to detect the presence of intruders in an organisation’s infrastructure, since any obstacle can potentially still be overcome by attackers (detection). Security is not merely a design, it is a continuous process where detection--and industry development--inform the next cycle of prevention and response measures.
  \item[Strict separation of tasks with competing aims] \hfill \\
In order that optimal decisions regarding the security design of an organisation be made, those responsible for security design and system administrators should have strictly different tasks. This can be implemented on various levels, from job descriptions and task agreements to formal Role Based Access Control systems with static separation of duty. The core of this lesson is that competing aims in an organisation should not be mediated by a single individual or group because that leads to bad compromises for both.
\end{description}

\centering{\textbf{Technical Lessons for All Organizations}}
\begin{description}
  \item[Strong patch management] \hfill \\
As time passes, vulnerabilities in software become known. This is true for any piece of software of nontrivial size. It is crucially important that software be kept at the latest available version. In order to facilitate this task, the person responsible for updates are advised to subscribe to security mailing lists and to configure (often built-in) patching frameworks.
  \item[Perform regular penetration tests] \hfill \\
Eric S. Raymond makes the case that given enough eyeballs, all bugs are shallow \cite{raymond1999cathedral}. Although he applied this concept to software development as a whole it may be applied to testing the strength of a security design, too. Regular penetration tests can help uncover weaknesses that are overlooked by the infrastructure’s designers and maintainers. Outside teams are not stuck in the same paradigms of the designers and are freer to think and act as an attacker would.
  \item[Monitor all systems] \hfill \\
System monitoring helps ensure the correct functioning of systems’ functionality and security. It is especially important to regularly make sure that detective measures continue to function and generate appropriate alerts with the responsible employees.
  \item[Separate vital logging services from the vital systems generating the logs] \hfill \\
This lesson is a more specific and technical version of the separation of duties and simultaneously recognizes the need for auditing. Logs cannot be part of the systems that generate them, since their trustworthiness would be compromised along with that system. By contrast, a separate logging service presents a very small attack surface when configured correctly and allows auditors to review a systems historical operation with greater confidence.  \item[Ensure forensic readiness] \hfill \\
The final lesson in the Black Tulip report sums up and combines others. It advocates a proactive stance within security teams: coordination the reception and processing of data, formulating a pre-planned incident response, incorporating feedback into future designs et cetera. In short, it recommends the implementation of a Plan-Do-Check-Act cycle and the mindset that goes along with it.
\end{description}

\centering{\textbf{Technical Lessons for CAs and Similar Organizations}}
\begin{description}
  \item[Harden all systems] \hfill \\
The default system settings are, per definition, not tailored to the situation in any specific organisation. As an implementation of the principle of least privilege administrators should disable any unnecessary daemons and customize configuration files to suit the needs at hand. As part of this, implementors should minimize the privileges of the daemons left and make sure that mission-critical machines are not overloaded to perform secondary tasks.
  \item[Correlate OCSP responder data to certificate serial numbers] \hfill \\
In the case of DigiNotar, an important early warning could have been OCSP requests for rogue certificates by users’ browsers. Whenever a browser performs an OCSP status lookup for a rogue certificate it represents an immediate opportunity to detect unauthorized certificates. More generally this is a lesson in utilizing and correlating as many data points as possible because it allows organisations to enhance their situational awareness.
  \item[Air gap vital systems] \hfill \\
Air gaps between sensitive servers and untrusted networks provide a strong form of access control that limits interaction with secure systems to properly credentialed personnel. This reduces the system’s attack surface. Static separation of duty (discussed above) can further increase security by allocating physical access from system access to different roles.
\end{description}

\subsection{D. Relating Lessons to ISO 27002 Chapters}

\begin{table}[h]
\begin{tabular}{| m{6cm} | m{10cm} |}
 \hline
 \multicolumn{1}{|c|}{\textbf{DigiNotar lesson}} & \multicolumn{1}{|c|}{\textbf{ISO-IEC 27002-2005 Chapter(s)}} \\
 \hline
Complement prevention with detection & 10.4.1 “Controls against malicious code” and 10.10.2 “Monitoring system use”\\ 
 \hline
Strict separation in tasks with competing aims & 10.1.3 “Segregation of duties”\\
 \hline
Air gap vital systems & 11.4.5 “Segregation in networks”\\
 \hline
Update all software products & 9.2.4 “Equipment maintenance” and 12.6.1 “Control of technical vulnerabilities” \\  
 \hline
 Harden all systems & 12.1.1 “Security requirements analysis and specification” and 11.1.1 “Access control policy” \\
 \hline
 Perform regular penetration tests & 15.2.2 “Technical compliance checking” \\  
 \hline
 Monitor your systems and network & 10.10.2 “Monitoring system use” and \\  
 \hline
 Correlate OCSP responder data to certificate serial numbers & 10.10.1 “Audit logging” and 10.10.5 “Fault logging” \\  
 \hline
 Separate vital logging services from the vital systems generating the logs & 15.3.2 “Protection of information systems audit tools” and 10.10.3 “Protection of log information” \\  
 \hline
 Ensure forensic readiness & 13 “Information security incident management” \\
 \hline
\end{tabular}
\caption{Relating the lessons from Fox-IT’s Black Tulip report to ISO 27001-2005 chapters.}
\label{table:1}
\end{table}

\bibliography{Assignment1.bib}
\bibliographystyle{splncs03}
\end{document}