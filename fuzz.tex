Device drivers implementing 802.11p are a promising source of undiscovered bugs. In particular, they are typically written in C code where it is easy to make mistakes and they may be audited less frequently than mainline kernel codes \cite{butti2008discovering}. These factors combined with the recent adoption of the 802.11p standard suggest insecure implementations exist. Fuzzing is an ideal approach to discover such vulnerabilities as it provides good results for the required investment as compared to a more tedious code review.\\

Thus to aid our search for undiscovered vulnerabilities we will design a fuzzing architecture. Design considerations will include the selection of an appropriate hardware configuration, of software tool(s) (i.e. leveraging existing tools such as Scapy, develop custom tools or leveraging tools used in Chapter \ref{sec:existing_attacks}) and of the 802.11p protocol fields to fuzz.