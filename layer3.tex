
Authentication is an extremely important property for the proper operation of VANETs. For example, an attacker could
inject false information into the network by announcing a non-existent traffic jam or a false accident report. A false traf- fic jam announcement could cause traffic to be diverted from one road to another and actually cause a traffic jam. A false accident announcement could cause emergency braking and potentially result in real accidents \cite{toor2008vehicle}.


Replay\\
Relay aka Wormhole \cite{raya2007securing} \\
Forged messages\\
Masquerading \cite{laurendeau2006threats}\\

Privacy is a major issue in VANETs because tracking vehicles would be easy and cost effective unless proper steps are taken. Attackers could install a network of cheap radio transceivers to eavesdrop on all wireless communication in the VANET. The larger the number of transceivers, the greater the strength. Then, vehicles could be linked to the actual identity of the person by tracking the movement pattern from home to workplace \cite{toor2008vehicle}
Many papers on this e.g. \cite{dotzer2006privacy} but they mainly propose solutions.\\


Insider/Outsider Eavesdropping \cite{laurendeau2006threats}\\


Sybil Attack - malicious vehicle pretends to be multiple other vehicles \cite{zhou2007privacy}\\

Bush telegraph - adding incremental errors to the information at each hop\cite{raya2007securing}\\

