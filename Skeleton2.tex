\documentclass[conference]{IEEEtran}

\usepackage{verbatim}	%Multi line comment using \begin{comment} and \end{comment)

\title{802.11p Literature Review and Attacks}
\author{\IEEEauthorblockN{Aniket Chaudhari and Alvin Cai and Wouter de Groot and Erik Schneider}
\IEEEauthorblockA{Technische Universiteit Eindhoven\\
Eindhoven, The Netherlands\\
\\
September 30, 2014}}

\begin{document}
\maketitle

\begin{abstract}
In this literature review, we describe 802.11p, 1609.2 and attacks on vehicular networks in the context of “Security and Privacy in Mobile Systems”.
\end{abstract}

\section{Introduction}

\begin{comment}
I think we should amend the first paragraph to provide an overview of VANET in general before jumping into 802.11p.

I made some amendments to para 2 and 3, to align to the modified scope. i.e. 
a) The literature review is the main meat of this project
b) It must provide a comprehensive survey of different attacks 
c) We can explain the attacks we want to implement in more detail
d) Focus on attacks in OSI layer 1-3
\end{comment}

Communications and netwoking are key factors in the development of an intelligent transportation systems which aims to automate the interactions among vehicles and infrastructure to achieve high levels of safety, efficiency and comfort. The IEEE has developed a system architecture known as wireless access in vehicular networks (WAVE) which comprises the IEEE 802.1p and IEEE 1609.x standards.

802.11p is an amendment to the 802.11 wireless communication standard specifically designed for vehicle to vehicle communication mainly related to safety issues such as warning of incidents. It utilizes the 5.9 GHz frequency band and is designed to accommodate very short-lived communication links. 802.11p is a PHY and MAC layer standards. The IEEE 1609 family of standards are a set of higher layer protocols which aims to address other issues such as security and efficient routing protocols.\\

In this paper we will provide a literature review of 802.11p, the security component of IEEE 1609 and provide an overview of vehicular network attacks targeted at the OSI layers of 1, 2 and 3. This survey will inform our choice of attacks to implement during the practical phase of the project. In addition, we will attempt to demonstrate new attacks through the implementation of a fuzzing framework.

\section{Wave Architecture}

A WAVE system consists of onboard units (OBUs) mounted in vehicles and stationary roadside units (RSUs) which may be installed in roadside infrastructure such as light poles. In this section we focus on WAVE features which support road safety applications such as cooperative driving and collision avoidance. These applications impose stringent requirements on message latency, authentication and a need to maintain communication across multiple hops \cite{raya2007securing} \cite{uzcategui2009wave}.

\subsection{IEEE 802.11p}

The IEEE 802.11p spectrum is structured into seven 10 MHz wide channels which comprise a control channel (CCH) and six service channels (SCH). The CCH is used for system control and safety purposes, two SCHs are reserved for safety purposes and the remaining four SCH are available for both safety and non-safety usage \cite{jiang2008ieee}. This highlights the importance IEEE places to provide availability to safety messages. Other main differences in the PHY layer serve to improve delay spread and minimise cross channel interference which we will not elaborate.\\

At the MAC layer, a major modification was to enable instantaneous data exchanges e.g. when two high speed vehicles in opposite directions pass each other. The original 802.11 protocol first involves a wireless station receiving a beacon from an access point which contains the wireless LAN properties such as data rates and security. Then, a series of message exchanges involving authentication and association occur before a basic service set (BSS) is established. It is only after BSS establishment does actual data exchange occur. \\

Each IEEE 802.11 data frame includes the BSSID within its address field. The BSSID is unique to each BSS and is used to filter messages which do not belong to that BSS. 802.11p preserves the data frame format but does away with BSS establishment. Instead, radios can directly transmit and receive data frames with the wildcard BSSID without additional overheads \cite{jiang2008ieee}.

802.11p radios can also organize themselves into WAVE basic service sets (WBSSs) similar to the traditional BSS, with the standard BSSID filtering rules. Nevertheless for safety reasons, radios can still receive packets sent with the wildcard BSSID (which are outside the WBSS).

\subsection{IEEE 1609.x}

IEEE 1609.x is a family of four different standards. Some of these standards are specific to a specific OSI layer; For example, 1609.4 defines a sub-layer in OSI layer two to control multichannel operation which is required as 802.11p only describes communications over an individual channel. Wave security services are described in IEEE 1609.2 and is not tied to any specific OSI layer. 1609.2 will be the main focus in this section.

1609.2 describes three main mechanisms \cite{ieee16092}: 
\begin{enumerate}
	\item Secure data exchange which involves signing and encrypting messages before sending and verifying and decrypting messages on reception
	\item Generating secure WAVE service advertisements (WSA) and verifying signed WSAs on reception
	\item Security processes such as generating a certificate request, verifying a certificate revocation list and other security management issues such as private key storage and the process flow to invoke security services.
\end{enumerate}

The standard also describes the public key infrastructure requirements and defines the supported cryptographic protocols. It should be noted that this document does \textit{not} support anonymous authentication.


\subsection{Equivalent standards}
Different vehicular networking projects, architectures and protocols exist in USA, Japan and Europe. Each geographical region has different regulations and an emphasis on different problems. The IEEE 1609 standards has mainly been the result of research and development activities in the USA. In EU, the results of such activities are contributing to the ETSI (European Telecommunications Standards Institute) ITS and ISO (International Organization for Standardization) CALM (Continuous Air-interface Long and Medium range) standardization. Nevertheless, a common factor across these regions is the usage of IEEE 802.11p  as the common data link protocol \cite{karagiannis2011vehicular}.




\section{Attacks}
Aijaz et al. \cite{aijaz2006attacks} employ attack trees to assess the vehicular communication threat model and provides a useful classification system for both existing and future attacks. 

We will describe the main attacks proposed and evaluate if the current standard are vulnerable to these attack. 



\subsection{PHY Layer}
For a packets in packets attack \cite{goodspeed2011packets}, an attacker injects specially crafted packets into the network that appear to be valid. However, upon radio interference, the outer packet is damaged and the receiver is tricked into accepting the inner packet that contains a malicious payload. This attack is perhaps extra plausible in a VANET scenario because we have trusted hardware and firmware components but maybe untrusted applications e.g. infotainment system?\\

Spoofing messages and the countermeasure of using double / directional antennas to detect spoofing


\subsection{MAC Layer}
Denial of service attacks can be performed by the aggressive injection of dummy messages over a network or by halting access to shared network devices. e.g. flooding, jamming, synch \cite{biswas2012ddos}\\



Black hole -  A black hole is formed by nodes which fail to propagate messages \cite{laurendeau2006threats} \\


Hidden vehicle - the hidden vehicle attack consists in deceiving vehicle A into believing that the attacker is better placed for forwarding the warning message, thus leading to silencing A and making it hidden, in DSRC terms, to other vehicles \cite{raya2007securing}\\




\subsection{Network Layer}

Authentication is an extremely important property for the proper operation of VANETs. For example, an attacker could
inject false information into the network by announcing a non-existent traffic jam or a false accident report. A false traf- fic jam announcement could cause traffic to be diverted from one road to another and actually cause a traffic jam. A false accident announcement could cause emergency braking and potentially result in real accidents \cite{toor2008vehicle}.


Replay\\
Relay aka Wormhole \cite{raya2007securing} \\
Forged messages\\
Masquerading \cite{laurendeau2006threats}\\

Privacy is a major issue in VANETs because tracking vehicles would be easy and cost effective unless proper steps are taken. Attackers could install a network of cheap radio transceivers to eavesdrop on all wireless communication in the VANET. The larger the number of transceivers, the greater the strength. Then, vehicles could be linked to the actual identity of the person by tracking the movement pattern from home to workplace \cite{toor2008vehicle}
Many papers on this e.g. \cite{dotzer2006privacy} but they mainly propose solutions.\\


Insider/Outsider Eavesdropping \cite{laurendeau2006threats}\\


Sybil Attack - malicious vehicle pretends to be multiple other vehicles \cite{zhou2007privacy}\\

Bush telegraph - adding incremental errors to the information at each hop\cite{raya2007securing}\\



Black hole -  A black hole is formed by nodes which fail to propagate messages \cite{laurendeau2006threats} \\


Hidden vehicle - the hidden vehicle attack consists in deceiving vehicle A into believing that the attacker is better placed for forwarding the warning message, thus leading to silencing A and making it hidden, in DSRC terms, to other vehicles \cite{raya2007securing}\\


\section{Methodology}
\subsection{ Implementations of Existing Attacks}
\label{sec:existing_attacks}
Ath5k is a common open source driver for Qualcomm Atheros based wireless chipsets in Linux and will be the focus for replicating existing attacks. Specifically, we will interface with the drivers for the packets in packets attack and modify the drivers for the DOS attack. We will then evaluate if the 802.11p reference devices (available at Twente) or Atheros drivers are vulnerable. If the drivers are vulnerable, we will attempt to propose a fix and evaluate the efficacy of this fix.\\

\subsection{Fuzzing Framework}
Device drivers implementing 802.11p are a promising source of undiscovered bugs. In particular, they are typically written in C code where it is easy to make mistakes and they may be audited less frequently than mainline kernel codes \cite{butti2008discovering}. These factors combined with the recent adoption of the 802.11p standard suggest insecure implementations exist. Fuzzing is an ideal approach to discover such vulnerabilities as it provides good results for the required investment as compared to a more tedious code review.\\

Thus to aid our search for undiscovered vulnerabilities we will design a fuzzing architecture. Design considerations will include the selection of an appropriate hardware configuration, of software tool(s) (i.e. leveraging existing tools such as Scapy, develop custom tools or leveraging tools used in Chapter \ref{sec:existing_attacks}) and of the 802.11p protocol fields to fuzz.

\section{Results}
This section will contain a list of exploits discovered as a result of our fuzzing. For each flaw, we will detail the method and the potential consequences of its exploitation.

\section{Analysis}
This section may be merged with Results if we discover a novel attack. Otherwise, this section will analyze the strength and limitations of our approach.

\section{Discussion}
This section will provide a critical assessment of the current research into 802.11p security, including our contributions. Among the lines of analysis to be considered are the following questions: How did our work contribute to the state of the art? What are the consequences for 802.11p? What are the most promising avenues of research? What are the limitations in our research?

\bibliographystyle{IEEEtran}
\bibliography{Skeleton.bib}

\end{document}