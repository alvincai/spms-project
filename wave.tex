
A WAVE system consists of onboard units (OBUs) mounted in vehicles and stationary roadside units (RSUs) which may be installed in roadside infrastructure such as light poles. In this section we focus on WAVE features which support road safety applications such as cooperative driving and collision avoidance. These applications impose stringent requirements on message latency, authentication and a need to maintain communication across multiple hops \cite{raya2007securing} \cite{uzcategui2009wave}.

\subsection{IEEE 802.11p}

The IEEE 802.11p spectrum is structured into seven 10 MHz wide channels which comprise a control channel (CCH) and six service channels (SCH). The CCH is used for system control and safety purposes, two SCHs are reserved for safety purposes and the remaining four SCH are available for both safety and non-safety usage \cite{jiang2008ieee}. This highlights the importance IEEE places to provide availability to safety messages. Other main differences in the PHY layer serve to improve delay spread and minimise cross channel interference which we will not elaborate.\\

At the MAC layer, a major modification was to enable instantaneous data exchanges e.g. when two high speed vehicles in opposite directions pass each other. The original 802.11 protocol first involves a wireless station receiving a beacon from an access point which contains the wireless LAN properties such as data rates and security. Then, a series of message exchanges involving authentication and association occur before a basic service set (BSS) is established. It is only after BSS establishment does actual data exchange occur. \\

Each IEEE 802.11 data frame includes the BSSID within its address field. The BSSID is unique to each BSS and is used to filter messages which do not belong to that BSS. 802.11p preserves the data frame format but does away with BSS establishment. Instead, radios can directly transmit and receive data frames with the wildcard BSSID without additional overheads \cite{jiang2008ieee}.

802.11p radios can also organize themselves into WAVE basic service sets (WBSSs) similar to the traditional BSS, with the standard BSSID filtering rules. Nevertheless for safety reasons, radios can still receive packets sent with the wildcard BSSID (which are outside the WBSS).

\subsection{IEEE 1609.x}

IEEE 1609.x is a family of four different standards. Some of these standards are specific to a specific OSI layer; For example, 1609.4 defines a sub-layer in OSI layer two to control multichannel operation which is required as 802.11p only describes communications over an individual channel. Wave security services are described in IEEE 1609.2 and is not tied to any specific OSI layer. 1609.2 will be the main focus in this section.

1609.2 describes three main mechanisms \cite{ieee16092}: 
\begin{enumerate}
	\item Secure data exchange which involves signing and encrypting messages before sending and verifying and decrypting messages on reception
	\item Generating secure WAVE service advertisements (WSA) and verifying signed WSAs on reception
	\item Security processes such as generating a certificate request, verifying a certificate revocation list and other security management issues such as private key storage and the process flow to invoke security services.
\end{enumerate}

The standard also describes the public key infrastructure requirements and defines the supported cryptographic protocols. It should be noted that this document does \textit{not} support anonymous authentication.


\subsection{Equivalent standards}
Different vehicular networking projects, architectures and protocols exist in USA, Japan and Europe. Each geographical region has different regulations and an emphasis on different problems. The IEEE 1609 standards has mainly been the result of research and development activities in the USA. In EU, the results of such activities are contributing to the ETSI (European Telecommunications Standards Institute) ITS and ISO (International Organization for Standardization) CALM (Continuous Air-interface Long and Medium range) standardization. Nevertheless, a common factor across these regions is the usage of IEEE 802.11p  as the common data link protocol \cite{karagiannis2011vehicular}.

